\chapter*[Introdução]{Introdução}
\addcontentsline{toc}{chapter}{Introdução}

A Organização das Nações Unidas (ONU) divulgou em 2015 o seu relatório referente às perspectivas de crescimento populacional, que constatou uma população mundial de 7,3 bilhões na metade do ano de 2015, projetando ainda, para 2030, um aumento para 8,5 bilhões (ONU, 2015). Esse aumento populacional representa também um aumento significativo na demanda por alimentos, energia e infraestrutura. A indústria agrícola tem um papel essencial ao satisfazer essa demanda por alimentos, de forma que o aumento de produtividade e eficiência é essencial.

A utilização de tecnologias para o aperfeiçoamento do processo produtivo e diminuição de custos já é há muito conhecida. Segundo RAMIZ (1988), “os custos de produção podem variar por diversos motivos. Pode-se destacar a utilização intensiva ou não de tecnologia”. Uma das soluções propostas é a utilização de robôs automatizados (também conhecidos como rovers) para a análise do solo e de suas propriedades e necessidades, tornando customizado o uso de insumos para o processo produtivo.

