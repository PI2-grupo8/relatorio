\chapter[Metodologia]{Metodologia}

\section{Gerenciamento}

A fazer
 %como por exemplo \lq\lq disquete de  3$\nicefrac{1}{2}$ polegadas\rq\rq. 

\subsection{Estrutura Analítica do Projeto (EAP)}

Tendo como objetivo a visão geral sobre as atividades que serão realizadas no decorrer do projeto, foi feita uma organização sistematizada das informações, sendo que a forma escolhida para isso foi a EAP. A EAP consiste em uma organização das entregas a serem feitas em um formato de árvore, geralmente indo de tarefas mais gerais para tarefas mais específicas.

Para a construção da EAP, foram levados em conta as diferentes entregas a serem realizadas dentro do ciclo de vida do projeto, assim como as áreas de atuação dentro da equipe. Além disso, é possível observar o alinhamento da EAP com as atividades previstas no cronograma e com os requisitos estabelecidos para o projeto.

A figura X é a representação gráfica da EAP do projeto.

Figura X. Estrutura Analítica do Projeto.

\subsection{Alocação dos recursos humanos}

A equipe do projeto, formada por 15 (quinze) integrantes, foi subdivida em 4 (quatro) áreas de atuação, que são elas: "Controle e Sensoriamento", "Localização no Ambiente e Informações", "Perfuração e Coleta" e "Tração, Energia Utilizada e Estrutura". As áreas são supervisionadas e administradas por uma gestora geral, Jéssica Guimarães, e por um gestor de qualidade, Leonardo Cambraia.

Cada uma das áreas acima citadas é responsável pela análise de alternativas para a solução e a escolha de uma destas para ser aplicada no projeto.

A figura X ilustra a estrutura de alocação de recursos humanos do projeto. A divisão dos integrantes foi feita levando-se em conta o interesse de cada um nas áreas propostas e o conhecimento prévio de cada um. É importante salientar que toda a estrutura está sujeita a mudanças, sempre visando o suprir bem as necessidades do projeto.

Figura X. Alocação de recursos humanos.

\subsection{Comunicação}

O sucesso de todo projeto é diretamente relacionado ao engajamento da equipe, sendo que para isso é necessária boa comunicação entre os membros da equipe. A tabela X detalha os métodos de comunicação utilizados pela equipe.

\textit Tabela a ser inserida

\subsection{Tempo}

Para a definição das atividades a serem realizadas durante o projeto utilizou-se como base os pacotes de trabalho estabelecidos na Estrutura Analítica do Projeto (EAP), onde os mesmos formam devidamente decompostos.

Com base nos dois grandes marcos do projeto referentes às releases 1 e 2, onde primeira abordará uma metologia de desenvolvimento tradicional (02/09/2016) e a segunda ágil (0/12/2015). Para tal feito, a equipe, tanto de gerência quanto de desenvolvimento precisará cumprir as atividades elucidadas no cronograma.

Após decompostos os pacotes de trabalho da EAP, a equipe de gerência reuniu-se para discutir como suas atividades seriam executadas, visando tanto uma paralelização de atividades quanto o tempo estimado e os recursos necessários para tal.

Para a determinação do tempo foi utilizado foram utilizadas as técnicas de \textbf{Analogia} e \textbf{Decisão em Grupo}, as quais, segundo o PMBOK (2012), representam:

\begin{itemize}
	\item Analogia: baseia-se em pacotes de trabalho/atividades similares de projetos anteriores para estimar a duração dos pacotes de trabalho e/ou atividades do seu projeto atual.
	\item Decisão em Grupo: nessa técnica o envolvimento da equipe de projeto nas estimativas proporcionam maior comprometimento da mesma com as atividades a serem realizadas.
\end{itemize}

Para padronização do controle de avanço físico das atividades do projeto, estabeleceu-se o seguinte critério:

\begin{table}[!htb]
	\centering
	\caption{Padronização do controle de avanço físico.}
	\begin{tabular}{|l|l|}
		\hline
		\textbf{Status}                                                   & \textbf{Atividade}                                                                                                                                                                                                                                                                                                                            \\ \hline
		\begin{tabular}[c]{@{}l@{}}0\%\end{tabular} & Não iniciada.                                                                                                                                                                                                                                                                                     \\ \hline
		\begin{tabular}[c]{@{}l@{}}10\%\end{tabular}     & Iniciada.                                                                                                                                                                                                                                                                                \\ \hline
		\begin{tabular}[c]{@{}l@{}}50\%\end{tabular}       & Na metade do trabalho.                                                                                                                                                                                                                                                                           \\ \hline
		\begin{tabular}[c]{@{}l@{}}100\%\end{tabular}       & Concluída.                                                                                                                                                                                                                                                                           \\ \hline
	\end{tabular}
\end{table}

Para o controle visual do status das atividades do projeto, estabeleceu-se o seguinte critério:

\begin{table}[!htb]
	\centering
	\caption{Controle visual do status.}
	\begin{tabular}{|l|l|}
		\hline
		\textbf{Cor}                                                   & \textbf{Atividade}                                                                                                                                                                                                                                                                                                                            \\ \hline
		\begin{tabular}[c]{@{}l@{}}Sem cor\end{tabular} & Não iniciada.                                                                                                                                                                                                                                                                                     \\ \hline
		\begin{tabular}[c]{@{}l@{}}Amarelo\end{tabular}     & Iniciada.                                                                                                                                                                                                                                                                                \\ \hline
		\begin{tabular}[c]{@{}l@{}}Vermelho\end{tabular}       & Atrasada.                                                                                                                                                                                                                                                                           \\ \hline
		\begin{tabular}[c]{@{}l@{}}Verde\end{tabular}       & Concluída.                                                                                                                                                                                                                                                                           \\ \hline
	\end{tabular}
\end{table}


\section{Riscos}

A probabilidade e os impactos do riscos para o projeto classificam-se conforme a tabela abaixo:

inserir tabela

O impacto dos riscos foram divididos em 3 níveis, onde:

\begin{itemize}
	\item Pequeno: consiste em um risco que afetará pouco o projeto, sua ocorrência não deve interferir no cronograma ou no custo do projeto.
	\item Regular: consiste em um risco que afetará consideravelmente o andamento do projeto, podendo atrasar datas referentes aos sub-processos de um pacote de trabalho no cronograma, as quais refletirão diretamente no custo do projeto.
	\item Alto: consiste em um risco de alto impacto ao projeto, sua ocorrência pode atrasar entregas de um ponto de controle, comprometer completamente um pacote de trabalho ou até mesmo a interação entre a equipe. Estratégias de prevenção deverão ser realizadas imediatamente.
\end{itemize}

Para uma maior visualização do quão crítico pode ser um risco, realizou-se a Matriz de Probabilidade e Impacto:

Inserir tabela

Segundo Scofano (2011), as estratégias para resposta de ameaças à um projeto podem ser classificadas da seguinte maneira:

\begin{itemize}
	\item Evitar: modificar o plano de projeto para eliminar o risco, a condição ou proteger os objetivos do projeto destes impactos. Mesmo sabendo que não existe a possibilidade de se eliminarem todos os eventos de risco, alguns riscos específicos, porém, podem ser evitados.
	\item Transferir: alterar a consequência de um risco para uma terceira parte juntamente com a responsabilidade de resposta. Essa estratégia não elimina o risco, apenas transfere sua responsabilidade para outra parte.
	\item Mitigar: reduzir a probabilidade e/ou consequências de um evento de risco adverso para um nível aceitável. Procura realizar ações preventivas para que não tenha consequências a serem reparadas.
	\item Aceitar: incluir um plano de contingência a ser executado quando da ocorrência de um. Esta técnica indica que a equipe não fará alterações no plano do projeto para negociar com um risco ou que não há resposta adequada.
\end{itemize}

Além das ameaças, também existem as estratégias para eventos que possam ser benéficos ao projeto, as quais são:

\begin{itemize}
	\item Explorar: utilizar para riscos com impacto positivo, caso exista a necessidade de garantir a oportunidade pela organização.
	\item Compartilhar: alocar integral ou parcialmente a propriedade da oportunidade a um terceiro que detenha maior capacidade de gerir determinada oportunidade para benefício do projeto.
	\item Melhorar: empregar para aumento de probabilidade e/ou impactos positivos de uma possível oportunidade.
	\item Aceitar: aproveitar uma devida oportunidade sem que haja alocação de esforços dos recursos do projeto.
\end{itemize}

Para a identificação dos riscos foram realizados brainstormings entre a equipe, objetivando uma maior transparência e alertas sobre os riscos em si. Os riscos do projeto encontram-se na tabela à seguir:

Inserir tabela
\begin{comment}
\begin{table}[h]
	\centering
	\label{tab01}
	
	\begin{tabular}{ccc}
		\toprule
		\textbf{Processing type} & \textbf{Property 1} (\%) & 
		\textbf{Property 2} $[\mu m]$ \\
		\midrule
		Process 1 & 40.0 & 22.7 \\
		Process 2 & 48.4 & 13.9 \\
		Process 3 & 39.0 & 22.5 \\
		Process 4 & 45.3 & 28.5 \\
		\bottomrule
	\end{tabular}

	\caption{Propriedades obtidades após processamento}
\end{table}
\end{comment}

\chapter[Requisitos]{Requisitos}

\section{Configuração}

O robô deverá permitir a inserção do número de coletas a serem feitas.
O robô deverá permitir a inserção da distância máxima a ser percorrida.

\section{Movimentação}

\begin{itemize}
	\item O carro deverá se movimentar entre as fileiras de morangueiros sem colisão;
	\item Ele também deve suportar as irregularidades dos solo sem problemas;
	\item Conseguir parar nos pontos de medição; 
	\item Manter uma velocidade adequada.
\end{itemize}

\section{Estrutura e Tração}

\begin{itemize}
	\item Dimensões : as dimensões do carro devem ser suficientes para acomodar os componentes internos do projeto e limitada a uma largura que o carrinho possa se movimentar, umas vez que as fileiras possuem uma largura especifica de acordo com cada plantação..
	\item Material: para a construção deste projeto, precisa-se de materiais específicos para cada parte do carrinho automático. Para a estrutura ( eixo, roda, eixo de transmissão, treliça) é necessário um material que seja maleável, leve, resistente à corrosão e resistente à esforços. Para o revestimento o material usado deve apresentar alta resistência. 
	\item Peso: para o projeto, o payload (componentes internos) será muito importante, logo  influenciará no projeto de estabilidade e no dimensionamento do motor de tração.
\end{itemize}

\section{Perfuração e Coleta}

\begin{itemize}
	\item O diâmetro do furo: neste caso, para definir a dimensão do diâmetro do furo, precisa-se determinar a largura do sensor que será inserido no furo. 
	\item Profundidade do furo: conforme as características  pesquisadas a respeito do sistema radicular do morangueiro e da plantação, a profundidade do furo não precisa ser muito longa, pois o furo será lateral.
	\item Potência da perfuratriz: Devido a textura areno-argilosa do solo no qual é feito o plantio do morango a perfuratriz deve gerar uma potência capaz de penetrar o solo. 
	\item Material da lâmina: deve ser muito resistente e helicoidal.
	\item Como estender o sensor e a broca: o espaço interno onde o sensor e a broca vão se recolher deve comportar pelo menos 10cm a mais em relação à profundidade do furo.
\end{itemize}

\subsection{Coleta de dados}

A fazer

\subsubsection{Sensores}

Conforme citado anteriormente, os dados relevantes para o projeto são a umidade do solo, a umidade relativa do ar e a temperatura do ambiente. Os sensores necessários para isso são:

\begin{itemize}
	\item Umidade do solo (higrômetro): O sensor deve variar o sinal de saída em função da umidade relativa do solo e ser compatível com Arduino;
	\item Umidade do ar/Temperatura:  O sensor deve variar o sinal conforme a variação da umidade do ar e temperatura, dentro dos limites de variação da temperatura ambiente, além de ser compatível com Arduino.
\end{itemize}

O robô será capaz de se locomover em linha reta autonomamente ao longo de uma certa distância, e para isso são necessários sensores de geolocalização (GPS) e sensores inerciais (acelerômetro e giroscópio):

\begin{itemize}
	\item GPS (Global Positioning System): O sistema deve adquirir o posicionamento do carro com boa precisão e passar essas informações para o sistema embarcado para a localização e informação da direção a ser tomada e para correção dos erros acumulados pelos sensores inerciaisl;
	\item Acelerômetro: O sensor deve ser capaz de adquirir as informações de aceleração linear em cada eixo para ser possível inferir o posicionamento imediato, e estes dados devem ser processados pelo Arduino;
	\item Giroscópio: O sensor deve ser capaz de adquirir as informações de rotação do carro para inferir a direção imediata e ser capaz de enviar dados para o processamento no Arduino.
\end{itemize}

\subsection{Transferência}

Os dados lidos pelos sensores e adquiridos a partir do Arduino deverão ser armazenados de forma a assegurar a confiabilidade dos mesmos. A integridade desses dados é de extrema importância considerando que uma medição incorreta pode acarretar em interpretações errôneas durante posteriores análises, impactando nas tomadas de decisão relativas à manutenção do solo.

\subsection{Apresentação dos dados}

O sistema deverá permitir a autenticação do usuário (login).
O sistema deverá permitir o upload e leitura do arquivo proveniente do robô.
O sistema deverá permitir a apresentação dos dados lidos para o usuário de forma gráfica.
O sistema deverá permitir a exportação dos dados lidos por meio de relatórios.
O sistema deverá permitir o armazenamento dos dados lidos.
O sistema deverá permitir a exclusão dos dados armazenados.

\chapter[Solução]{Solução}

\section{Tração, Energia e Estrutura}

\subsection{Tração}

\subsubsection{Análise comparativa}

\subsubsection{Sistema de esteira}

O sistema de tração por esteiras é largamente utilizado em aplicações agrícolas e em aplicações gerias nas quais é necessária a locomoção em terreno acidentado. O sistema tradicional de transmissão mecânica por esteiras é formado por diversos componentes, como conversor de torque, amortecedores e dumpers.

Este sistema possui como vantagem o alto fornecimento de tração ao veículo, mas requer um grande espaço para acoplamento ao veículo, além da baixa disponibilidade de materiais para sua fabricação. Outra desvantagem é que este sistema impede que o veículo realize manobra de contra-rotação.

\subsubsection{Sistema de roda}

É a configuração mais usada em todos os tipos de veículos, e literatura é bastante consolidada no que se refere a sistemas de transmissão e acoplamentos para veículos movidos sobre rodas. Este sistema foi o escolhido para implantação neste projeto devido à sua viabilidade de projeto e fabricação.

Foi realizada uma pesquisa de modelos de rodas disponíveis, tendo como requisito um diâmetro de aproximadamente 10cm, e pneus com perfil off-road, conforme mostra a imagem do link abaixo.  (INSERIR IMAGEM no LateX)

\subsubsection{Sistema aranha}

Este sistema de locomoção é bastante engenhoso. Consiste de vários braços, geralmente 6 ou 8, comandados por 2 motores ou mais em cada braço. Este sistema pode ser usado nos mais diversos tipos de terreno por fornecer alta  estabilidade e tração satisfatória. A cinemática envolvida na movimentação e no controle deste tipo de sistema é muito mais complexa do que nos sistemas de locomoção por esteira e por rodas, devido ao alto número de motores requeridos.

\subsubsection{Solução adotada para o projeto}

Sistema de Roda (Eixos / Suspensão/ Rodas)

\subsection{Energia}

\subsubsection{Análise comparativa}

\subsubsection{Motores Elétricos Aplicados para locomoção}

Três principais tipos de motores são aplicados para robôs, são eles: motor de passo, servo motor, e motor de corrente continua.

O principio de funcionamento dos motores de passo é transformar pulsos elétricos recebidos em variações angulares constantes e discretas. A variação dos pulsos elétricos controlam diretamente o sentido de rotação do seu eixo. Geralmente esses motores são aplicados em impressoras, scanners e equipamentos hospitalares (Solarbotics, 2001).  

Por sua vez, a estrutura  básica do servo motor consiste em um sistema de controle em malha fechada que converte um sinal elétrico num movimento de uma alavanca através de uma comparação entre a posição anterior e a desejada. Além disso, o sistema é capaz de determinar o sentido que o motor vai girar através da polaridade da tensão aplicada em PWM (Pulse Width Modular). Com o intuito de transferir mais torque para o eixo, o motor possui engrenagens que ajudam a reduzir a rotação do motor. São empregados em aeromodelismo, e outras aplicações de baixo torque requerido (NCB, 2016).

Por último, o motor de corrente contínua funciona através das forças de campo produzidas entre o campo magnético e a corrente de armatura (no rotor). Com um comutador, a variação da corrente proporciona uma corrente alternada no equipamento, gerando a rotação do eixo, e por fim, torque mecânico. Esse motor é aplicado desde pequenas aplicações até grandes, eles possuem elevada robustez, ampla variação de velocidade e baixa manutenção.

\subsubsection{Baterias}

Algumas especificidades devem ser levadas em conta na hora de optar pela escolha de uma bateria, pois ela deve suprir as demandas do projeto. As mais importantes são a carga elétrica (ou corrente) e a tensão (MAGALHÃES, 2011 e SILVA, 2011).

Os cálculos de autonomia para as baterias não são simples e necessitam de muitos parâmetros. Na literatura não há uma fórmula exata para calcular esses valores. Por isso são usadas diversas aproximações e parâmetros mais relevantes são determinados (MEGGLIARO, 2006).
Três parâmetros principais que podem ser escolhidos e demonstrados: curva de descarga, a capacidade da bateria e a capacidade de descarga (MEGGLIARO, 2006). 

A curva de descarga representa o decaimento da tensão ao longo do consumo da capacidade nominal. O segundo parâmetro é a capacidade da bateria, o qual quantifica o tempo para que ocorra uma descarga total da bateria. Sua medida usual é Ah (Ampère x hora) (MEGGLIARO, 2006).  

A capacidade de descarga é quanto a bateria consegue fornecer sem que ocorra um superaquecimento dela, que poderia causar danos imensos a todo o projeto. Ela vem representada nas especificações pela letra C (MEGGLIARO, 2006).

\textbf{Tipos de Baterias}:

\begin{itemize}
	\item Bateria Íon- Lítio: Esse tipo de bateria é o mais utilizado atualmente no mercado para aplicações em aparelhos eletrônicos, tais como celulares, notebooks e tablets. Seu funcionamento consiste no uso de íons lítio presentes no eletrólito na forma sais dissolvidos em solventes não aquosos. Suas principais características são baixas taxas de autodescarga, longos ciclos de vida e segurança no manuseio(CASTILLO, 2002). Também possuem alta densidade de energia quando comparadas às baterias de hidreto metálico de níquel, as quais eram bastante utilizadas antes da inserção das baterias de íon- lítio no mercado.
	\item Bateria de Polímero de Lítio: Este tipo de bateria também utiliza o lítio, no entanto seus eletrólitos estão contidos em um polímero, diferentemente das baterias de íon-lítio, em que os eletrólitos estão contidos em solventes. As baterias de polímero de lítio são baterias de tecnologia ultrafinas. Isso é possível devido à sua maior densidade de energia quando comparada com as baterias de íon- lítio. A utilização dessas baterias visa atender à nova geração de computadores e aparelhos eletrônicos portáteis (Shneider, 2011).
\end{itemize}

\subsubsection{Solução e Sistema de Alimentação}

\subsubitem Motores

\subsubsectionmark{Motores}

\paragraph{Motores}

\paragraph{Componentes e ligações}

Ao dimensionar a carga, chega-se aos valores da Tabela a seguir:

Inserir tabela

Vale ressaltada que as correntes que passarão pelo driver serão as mesmas dos motores, como serão utilizados 2 drivers de 4 A cada, resulta em 12 A no total. Para dimensionamento da bateria necessária, estipula-se o que pode ser fornecida em decorrência das correntes máximas e do tempo desejado, por isso, optou-se por uma bateria de 9 Ah, o qual dará uma autonomia esperada de 0,75 horas de duração, ou seja,  45 minutos em funcionamento pleno, no pior cenário, onde todos os componentes estariam necessitando de correntes máximas.

\subsection{Estrutura}

\subsubsection{Análise comparativa}

\paragraph{Design}

\subparagraph{Design A}

\subparagraph{Design B}

\paragraph{Alocação dos componentes}

\subparagraph{Disposição A}

\subparagraph{Disposição B}

\paragraph{Materiais}

\subparagraph{Material A}

\subparagraph{Material B}

\subsubsection{Solução adotada para o projeto}

\paragraph{Design}

\paragraph{Alocação dos componentes}

\paragraph{Material}

\section{Perfuração e Coleta dos Dados}

\subsection{Análise Comparativa}

\subsubsection{Sensores de Umidade}

Sensor A

Sensor B

\subsubsection{Sistema de Perfuração}

Opção A

Opção B

\subsection{Solução Adotada para o Projeto}

\section{Localização no ambiente e Informações}

\subsection{Localização no ambiente}

\subsubsection{Análise comparativa}

\paragraph{Processamento de imagens}

\paragraph{Localização por meio de sensores/GPS}

\subsubsection{Solução adotada para o projeto}

\paragraph{Localização por meio de sensores/GPS}

\subsection{Informações}

\subsubsection{Análise comparativa}

\paragraph{Aquisição (das informações)}

\subparagraph{Umidade/Temperatura}

DHT11

\subparagraph{Umidade do solo}

Higrômetro (FC-28)

Opção B

\paragraph{Armazenamento dos dados}

Os dados serão armazenados no cartão de memória da Raspberry Pi em um arquivo no formato CSV (Comma Separated Value). Os dados serão dispostos em colunas sendo que a primeira coluna especifica a posição da medição em metros, a segunda coluna a umidade do solo, a terceira especifica a umidade relativa do ar e a quarta a temperatura ambiente.

Os dados dispostos desta maneira facilitam o processamento e análise dos dados pela base, assim como sua apresentação gráfica para o operador.

\paragraph{Leitura (das informações)}


\chapter[Integração da solução]{Integração da solução}

\chapter[Custos]{Custos}

\chapter[Considerações finais]{Considerações (finais)}

\section{PC1}

\subsection{Desafios}

\subsection{Resultados esperados}