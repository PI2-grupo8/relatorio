\chapter[Metodologia]{Metodologia}

\section{Gerenciamento}

A fazer
 %como por exemplo \lq\lq disquete de  3$\nicefrac{1}{2}$ polegadas\rq\rq. 

\subsection{Estrutura Analítica do Projeto (EAP)}

Tendo como objetivo a visão geral sobre as atividades que serão realizadas no decorrer do projeto, foi feita uma organização sistematizada das informações, sendo que a forma escolhida para isso foi a EAP. A EAP consiste em uma organização das entregas a serem feitas em um formato de árvore, geralmente indo de tarefas mais gerais para tarefas mais específicas.

Para a construção da EAP, foram levados em conta as diferentes entregas a serem realizadas dentro do ciclo de vida do projeto, assim como as áreas de atuação dentro da equipe. Além disso, é possível observar o alinhamento da EAP com as atividades previstas no cronograma e com os requisitos estabelecidos para o projeto.

A figura X é a representação gráfica da EAP do projeto.

Figura X. Estrutura Analítica do Projeto.
