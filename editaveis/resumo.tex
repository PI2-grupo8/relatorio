\begin{resumo}
% RESUMO ABAIXO
É um fato que a variabilidade de espaço e tempo influenciam nos rendimentos de diversas plantações.
Tendo isso em foco, o projeto visa o monitoramento de lavouras de morango, em alinhamento com conceitos de Agricultura de
Precisão, que integra informações do plantio com tecnologias atuais visando o gerenciamento mais detalhado do sistema de
produção agrícola em relação a todos os processos envolvidos na produção.
O foco principal do projeto, tendo como base o estudo de caso de um produtor local de morangos, será a medição de umidade do solo,
do ar e temperatura através de sensores pré-definidos, sendo que estas propriedades foram delimitadas em nosso
escopo devido a sua importância neste tipo de plantação.
% RESUMO ACIMA
 \vspace{\onelineskip}

 \noindent
 \textbf{Palavras-chaves}: Agricultura de Precisão, Coleta de Dados, Lavoura de Morango, Veículo Autônomo
\end{resumo}
