\section{Subsistema: Localização e Informações}
Este tópico descreve as soluções para localização do veículo no terreno da lavoura, sua trajetória e
como serão processados e armazenados os dados coletados das medições.

  \subsection{Localização}

  Pela necessidade do veículo realizar seu percurso de maneira autônoma, as soluções identificadas
  pelo grupo basearam-se em processamento de imagens, GPS e sensores de proximidade.

  Tendo em vista o curto período de tempo destinado ao desenvolvimento do projeto e a falta de
  conhecimento da equipe sobre processamento de imagens, essa opção apresentou um alto grau de risco
  e foi descartada.

  Já a opção do GPS se tornou insatisfatória para o projeto, uma vez que erros
  referentes à precisão podem acarretar em um comprometimento da plantação
  por parte do veículo.

  Dessa forma, foram feitos estudos sobre sensores de proximidade, especificamente
  ultrassom e infravermelho, os quais demonstraram ser uma solução viável e foram alvo de
  estudos mais aprofundados.

  Inicialmente, a opção baseada em ultrassom se mostrou mais adequada, porém foi possível perceber que os principais sensores
  comerciais que se baseavam nesse princípio possuíam uma faixa de detecção bastante re-
  duzida. Sendo assim, foi escolhido um sensor de proximidade baseado em infravermelho
  apresentado na figura \ref{fig:infrared}.

  \begin{figure}[!htbp]
  \begin{center}
  \includegraphics[width=.7\textwidth]{figuras/infrared.eps}
  \caption{\label{fig:infrared}Sensor de proximidade infravermelho.}
  \end{center}
  \end{figure}

  O princípio de funcionamento deste sensor é o princípio da triangulação.
  Um feixe de luz é emitido por um diodo laser ou um LED infravermelho.
  Ao ser refletido por um objeto, esse raio é detectado por um PSD
  (\textit{Position Sensing Device} -- Dispositivo de Monitoramento de Posição).
  De acordo com a distância do objeto que refletiu a luz, esse raio incide
  de modo diferente no PSD. A figura~\ref{fig:infrared_func} ilustra o
  princípio da triangulação.

  \begin{figure}[!htbp]
  \begin{center}
  \includegraphics[width=.7\textwidth]{figuras/infrared_func.eps}
  \caption{\label{fig:infrared_func}Princípio da triangulação.}
  \end{center}
  \end{figure}

  A escolha do elemento foi motivada pela capacidade de detecção
  de proximidade ser adequada para o funcionamento do algoritmo
  de lcomoção, mais especificamente entre 3 e 80cm.  Além disso,
  este elemento apresenta pouca interferência com a luz visível e
  possui um interfaceamento simples com um microcontrolador.

  \subsection{Algoritmo de Locomoção}
    Dado a utilização de sensores infravermelho e a disposição da lavoura de morango
    é possível identificar um típico problema denominado \textit{Wall Follower}.

    Segundo \cite{huang2009}, \textit{Wall Follower},
    também conhecido como regra da mão direita ou esquerda, é válido para percursos conexos, onde todas as paredes são ligadas entre si, mantendo a mão próxima a parede do labirinto em
    que se inclui, garantindo assim que o corpo não se perca e finalize
    o percurso. O principal problema desta regra ocorre quando um percurso
    possui loops de passagens que projetem uma trajetória sem fim.

    \begin{figure}[!htbp]
    \begin{center}
    \includegraphics[width=.7\textwidth]{figuras/wallfollower.eps}
    \caption{\label{fig:wallfollower}Exemplo de \textit{Wall Follower}.}
    \end{center}
    \end{figure}

    \vfill
    \pagebreak

    Basicamente o veículo utilizará as paredes dos canteiros como guia, realizando a
    coleta de dados sempre a sua direita. Para viabilizar essa solução serão necessárias algumas adaptações no terreno com a adição de estruturas à lavoura, criando uma espécie de circuito. A figura \ref{fig:ambientadapt} representa o caminho a ser seguido pelo veículo.

    \begin{figure}[!htbp]
    \begin{center}
    \includegraphics[width=.7\textwidth]{figuras/adapt.eps}
    \caption{\label{fig:ambientadapt}Adaptação do ambiente.}
    \end{center}
    \end{figure}

    \vfill
    \pagebreak

    Para a escolha dos componentes para a construção das barreiras ponderou-se
    necessidade de um material barato e de fácil instalação/manuseio.
    Desta forma, o plástico e a madeira foram os materiais escolhidos para compor os obstáculos
    que irão ajudar o veículo a percorrer o trajeto.

  \subsection{Informações}

  \subsubsection{Unidade de Processamento}

  Para processamento das informações foi
  escolhida foi o Raspberry Pi B+,
  É uma unidade robusta que conta com o microprocessador
  BCM2835 da Broadcom, com clock de
  700MHz e modo de baixo consumo, uma GPU Dual Core VideoCore
  IV\textregistered, SDRAM de 512MB, 4 portas USB, 1 porta RJ45
  (Ethernet), 1 porta HDMI e 40 pinos para GPIO~\cite{raspref}

  Esta unidade terá um sistema operacional Linux embarcado, responsável
  por gerenciar o armazenamento dos dados obtidos pelos sensores e
  processar decisões do controle do veículo.

  \begin{figure}[!htbp]
  \begin{center}
  \includegraphics[width=.7\textwidth]{figuras/raspberry.eps}
  \caption{\label{fig:raspberry}Raspberry Pi B+.}
  \end{center}
  \end{figure}

  \subsubsection{Armazenamento dos Dados}

  Para o armazenamento de dados foi escolhido um cartão de memória
  Micro SD de 8GB da Sandisk. Sua escolha foi motivada devido à
  necessidade de um dispositivo de armazenamento portátil, capaz de
  armazenar os dados adquiridos a partir dos sensores para a posterior
  análise. Além disso, a placa microprocessadora escolhida oferece
  suporte para cartão de memória, dispensando o uso de um módulo avulso
  para a realização das operações de armazenamento. A figura~\ref{fig:sdcard}
  apresenta o cartão de memória.

  \begin{figure}[!htbp]
  \begin{center}
  \includegraphics[width=.5\textwidth]{figuras/sdcard.eps}
  \caption{\label{fig:sdcard}Cartão de memória.}
  \end{center}
  \end{figure}

  Os dados serão armazenados no cartão de memória da Raspberry Pi
  em um arquivo no formato CSV (\textit{Comma Separated Value}). Os dados
  serão dispostos em colunas com a seguinte ordem: posição da medição
  em metros, umidade do solo, umidade relativa do ar e temperatura do
  ambiente.
  Os dados dispostos desta maneira facilitam o processamento e
  análise posterior, assim como sua apresentação gráfica para
  o operador.

  \subsubsection{Apresentação das Informações}

  Uma parte importante do projeto consiste na apresentação dos
  dados para o usuário encarregado pela gestão da lavoura. Para essa demanda,
  foi definido o desenvolvimento de uma aplicação web, utilizando
  o \textit{framework} Rails.

  A escolha de desenvolver utilizando-se essa linguagem junto a
  esse \textit{framework} foi feita com base na facilidade que essa
  tecnologia fornece para o desenvolvimento e com base na experiência
  da equipe responsável pelo desenvolvimento da aplicação, visto que
  todos os membros do grupo já possuem experiência com a tecnologia.

  Em concordância com os requisitos estabelecidos, essa aplicação
  permitirá o upload do arquivo CSV gerado pelo veículo, dentro de
  uma área restrita da aplicação, disponível apenas após a realização
  do login.
  Além da representação gráfica, o gestor da lavoura poderá obter
  esses dados em formato de relatório, sendo permitida sua
  exportação para um arquivo PDF (Portable Document Format).

  \vfill
  \pagebreak

  Após a obtenção dos dados, o usuário possuirá a opção de visualiza-los de forma gráfica, sendo que esta visualização se dará
  com base em um mapa de calor (\textit{heatmap}), conforme a figura \ref{fig:heatmap} representa, para uma melhor representação da lavoura.

  O conceito dos mapas de calor é baseado na representação de diversos pontos sobrepostos em um mapa, sendo que quanto mais pontos
  forem existentes em uma mesma posição, uma coloração mais forte será atribuída, por isso ganhando essa denominação.

  Para a implementação dos mapas de calor no contexto deste projeto, foi utilizada a API (\textit{Application Programming Interface})
  do Google Maps. Essa API permite acesso as imagens dos mapas e a posterior aplicação de uma camada com os pontos do mapa de calor.

  \begin{figure}[!htbp]
  \begin{center}
  \includegraphics[width=.5\textwidth]{figuras/heatmap.eps}
  \caption{\label{fig:heatmap}Exemplo de mapa de calor.}
  \end{center}
  \end{figure}
