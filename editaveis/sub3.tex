\section{Subsistema: Localização e Informações}

  \subsection{Adaptação do Ambiente}

  Sabendo-se que o plantio do morango utiliza o arranjo de fileiras,
  foram analisadas algumas possíveis soluções referentes à locomoção do veículo em tal ambiente.

 \begin{itemize}

 \item Processamento de Imagens

 Processamento de imagens consiste em uma técnica para a análise de dados
multidimensionais, que permite manipular e tratar imagens com objetivo de
obter informações e melhorar as características visuais da imagem.
A primeira etapa consiste em adquirir uma imagem, por exemplo,
através de uma câmera digital, de um scanner laser, de um ultrassom,
de um ressonador magnético, ou por qualquer outro meio. Após a captura,
 uma imagem precisa ser representada de forma apropriada para tratamento
 computacional, ou seja, sua representação se dará como uma função
 bidimensional ou tridimensional. Feito isso, inicia-se o pré-processamento
  da imagem. Nessa etapa a imagem passa por um processo de filtragem,
  no qual são eliminados os ruídos, que podem ser obtidos durante a
  captura da imagem, melhorando assim a qualidade e permitindo uma melhor
  discriminação dos objetos presentes na mesma.

 Pelo curto tempo destinado ao desenvolvimento do projeto e a falta de
 conhecimento da equipe sobre o assunto, a escolha por processamento de
  imagens acabou sendo descartada pelo seu alto grau de risco ao projeto.

 \item Localização por meio GPS

 O sistema de posicionamento global (GPS) é uma tecnologia
 para navegação em ambientes externos. O GPS foi desenvolvido
 pelo Departamento de Defesa Americano, sendo composto por um conjunto
 de satélites que transmitem um sinal de rádio-freqüência codificado.
 Usando-se métodos de trilaterização avançados, receptores no solo podem
  medir as suas posições baseado no tempo de viagem dos sinais de
  rádio-freqüência dos satélites, permitindo o cálculo da latitude,
  longitude e altitude do receptor.

O principal problema de se usar marcos naturais do ambiente na
navegação é detectar e combinar as características dos marcos
 a partir das informações dos sensores. A escolha natural do sensor
 para esta tarefa é a visão computacional. A maior parte dos marcos
 naturais do ambiente baseados em visão são longas bordas verticais,
como portas, junções de paredes e luzes no teto.

Pelo local da plantação ser um ambiente de tamanho de visibilidade
baixa e possuir características específicas da plantação, a localização
por GPS se torna insatisfatória para o projeto, uma vez que erros
referentes à precisão podem acarretar em um comprometimento da plantação
por parte do veículo.


\item Localização por Sensoriamento

As abordagens estudadas que podem resolver o problema da localização
basearam-se no sensoriamento infravermelho e ultrassônico.
Dado as condições da lavoura de morango ambos poderiam ser
aplicados como um problema referente à Wall follower.

Wall follower é a regra mais conhecida para percurso em labirintos,
também conhecido como regra da mão direita ou esquerda.
Ela é válida para percursos conexos, onde todas as paredes são
ligadas entre si, mantendo a mão próxima a parede do labirinto em
que se inclui, garantindo assim que o corpo não se perca e finalize o percurso. O principal problema desta regra ocorre quando um percurso possui loops de passagens que projetem uma trajetória sem fim.


\item \textit{Wall Follower}

A solução adotada para o projeto baseou-se na utilização de sensores
infravermelho, pois atendiam os requisitos com maior simplicidade,
menor custo técnico e computacional.

Todavia, devido as soluções adotadas para tração, movimentação das
rodas e perfuração (realizada somente pelo lado direito do veículo),
serão necessárias algumas modificações no ambiente de plantio para modelagem
do sistema como um cenário “Wall Follower” citado acima.

Para que o veículo seja capaz de percorrer todas as fileiras sem passar
 pelo mesmo caminho de maneira redundante, foi pensada a adição de
 estruturas à lavoura para montar uma espécie de circuito. O esquemático
  abaixo representa a estrutura da lavoura e o caminho a ser percorrido pelo
   veículo que irá seguir sempre a parede à sua direita até completar todo o
   percurso.

   Para a escolha do material ponderou-se necessidade de um material
   barato e de fácil instalação/manuseio. Desta forma, o plástico e
   a madeira foram os materiais escolhidos para compor os obstáculos
   que irão ajudar o veículo a percorrer o trajeto.

\end{itemize}

  \subsection{Sensoriamento: Localização}

  \subsection{Algoritmo de Localização}

  \subsection{Armazenamento dos Dados}

  \subsection{Leitura das Informações}
