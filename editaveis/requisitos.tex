\chapter{Requisitos}

  Tendo em vista a concepção e validação da solução, foram definidos os seguintes
  requisitos, sendo estes relativos aos subsistemas do projeto.

  \section{Mecânica e Alimentação}

    \begin{itemize}
      \item O veículo deverá permitir a movimentação em terrenos irregulares.
      \item O veículo deverá permitir a manutenção de uma velocidade constante e pré-determinada.
      \item O veículo deverá possuir dimensões capazes de acomodar os componentes internos definidos no projeto.
      \item O veículo deverá possuir uma largura inferior ao caminho pré-definido para sua movimentação.
      \item O veículo deverá possuir na composição de sua estrutura materiais maleáveis, leves e resistentes à corrosão e à esforços.
      \item O veículo deverá possuir em seu revestimento materiais de alta resistência.
      \item O veículo deverá possuir um peso suficiente para permitir a perfuração.
      \item O veículo deverá possuir um peso que não impossibilite sua movimentação em solos que não possibilitem uma boa tração.
      \item O veículo deverá possuir uma bateria com uma capacidade suficiente para suprir o gasto energético do motor e demais componentes.
      \item O veículo deverá possuir um motor com torque suficiente para garantir sua movimentação.
      \item O veículo deverá possuir uma autonomia suficiente para realizar a coleta de dados nos canteiros e retornar para a origem.
      \item O veículo deverá possuir um revestimento resistente a respingos de água.
    \end{itemize}

  \section{Perfuração e Coleta}

    \begin{itemize}
      \item O veículo deverá realizar a leitura e envio de dados referentes a umidade do solo, umidade relativa do ar e temperatura do ambiente nos pontos de medição definidos.
      \item O veículo deverá permitir a inserção do número de canteiros a serem percorridos.
      \item O sistema de perfuração deverá ser capaz de realizar uma perfuração no plástico protetor das raízes e introduzir o sensor de umidade no solo
      \item O sistema de perfuração deverá garantir a integridade do sensor.
    \end{itemize}

    \subsection{Localização e Informações}

      \begin{itemize}
        \item O veículo deverá realizar seu percurso.
        \item O veículo deverá permitir a inserção do número de coletas a serem feitas.
        \item O veículo deverá permitir a movimentação entre as fileiras de morangueiros sem colisão.
        \item O sistema de coleta de dados deverá possuir sensores de geolocalização e inerciais que auxiliem na movimentação autônoma.
        \item O sistema de apresentação dos dados deverá permitir a autenticação do usuário (login).
        \item O sistema de apresentação dos dados deverá permitir o upload e leitura do arquivo proveniente do veículo.
        \item O sistema de apresentação dos dados deverá permitir a apresentação dos dados lidos para o usuário de forma gráfica.
        \item O sistema de apresentação dos dados deverá permitir a exportação dos dados lidos por meio de relatórios.
        \item O sistema de apresentação dos dados deverá permitir o armazenamento dos dados lidos.
        \item O sistema de apresentação dos dados deverá permitir a exclusão dos dados armazenados.
      \end{itemize}
