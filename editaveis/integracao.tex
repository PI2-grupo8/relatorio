\chapter{Integração dos Subsistemas}
Devido a divisão do projeto nos subsistemas de tração, energia e estrutura;
 perfuração e coleta; localização no ambiente e infomações é necessário que haja uma proposta de
 integração dessas partes. Nessa proposta vamos tratar tração, energia e estrutura como módulo 1
 perfuração e coleta como módulo 2 e localização no ambiente e informações como módulo 3.
 Fica evidente, pelo curto tempo disponível, que os módulos devem começar a ser desenvolvidos simultaneamente.
 Porem, mais do que isso, é importante sincronizar atividades críticas e interdependentes tendo em vista a estrutura analítica
 do projeto.

 O módulo 1 alimenta o sistema como um todo e também recebe sinais de controle do módulo 3. Esses sinais são adquiridos
 pelos sensores de localização durante a rotina movimentação. Durante a rotina de perfuração o módulo 2 recebe sinais de
 de controle do módulo 3 e devolve o dado recolhido pelo sensor de umidade.

 Tendo em vista essa correlação, para que possamos realizar testes durante o desenvolvimento do projeto, os módulos 1 e 3
 devem manter ritimos proporcionais de progresso. O módulo 2 não retem outras atividades, já que só recebe dois tipos
 de sinal de controle e devolve a medição do sensor.
