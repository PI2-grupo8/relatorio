\chapter{Visão geral dos subsistemas}

A integração do sistema configura-se como fator decisivo para o cumprimento dos objetivos definidos.
Sendo assim, elaborou-se o esquemático, conforme figura \ref{fig:visao_geral}, com a finalidade de fornecer uma visão geral a
respeito dos principais aspectos de integração dos subsistemas.

\begin{figure}[!htbp]
\begin{center}
\includegraphics[width=.6\textwidth]{figuras/all_systems.eps}
\caption{\label{fig:visao_geral}Visão geral do sistema.}
\end{center}
\end{figure}

% Talvez quebrar a imagem em 3

O subsistema de Mecânica e alimentação é composto por tração, direção, suspensão, estrutura e alimentação. Os motores
responsáveis pela tração das rodas serão controlados por um driver, que por sua vez receberá sinais lógicos da placa Arduino
para a determinação do sentido de rotação e velocidade angular. Este driver será alimentado pela bateria do sistema conforme
figura \ref{fig:visao_geral} (diagrama unifilar). Em conjunto aos elementos de tração, os de suspensão e direção também serão integrados de forma
a compor o chassi do veículo. Por fim, os elementos mencionados serão integrados à estrutura, mais especificamente à carroceria.

Vale ressaltar que o projeto da carroceria deve considerar em seu espaço interno todos os componentes eletrônicos, de
alimentação e de perfuração.

Sendo assim, percebe-se que o subsistema de perfuração e coleta apresenta alto grau de integração com os demais subsistemas.
Tendo como núcleo central a placa Arduino, esse subsistema tem como objetivo realizar a coleta dos dados provenientes dos
sensores. Seu ponto crítico de integração consiste em projeto e alocação do sistema de perfuração, responsável por introduzir
o sensor de umidade de solo no ponto desejado sem causar danos ao mesmo. Todos os componentes eletromecânicos do subsistema,
também serão alimentados pela bateria.

%\vfill
%\pagebreak

Os dados adquiridos pelos sensores e lidos pela placa Arduino, serão enviados para a unidade de processamento, composta por
uma placa Raspberry Pi, um sistema operacional Linux embarcado, o qual executará os algoritmos de localização do sistema. Este
algoritmo irá tomar decisões a respeito da localização, e as enviará para o conjunto Arduino/Driver visando o controle de
direção. Além disso, esse algoritmo será responsável por armazenar os dados referentes à coleta em seus respectivos
pontos. Este armazenamento será implementado em um cartão de memória, que estará disponível para posterior visualização por
parte do usuário final.
